\documentclass[11pt,a4]{article}


% PACKAGES ----------
\usepackage[utf8]{inputenc} \usepackage{amsmath,amsfonts,amssymb}
\usepackage[left=2.5cm,right=2.5cm,top=2cm,bottom=2cm]{geometry}
\usepackage{xcolor} \usepackage{graphicx} \usepackage{hyperref}
\usepackage[lite]{mtpro2}


% MACROS ------
\newcommand{\ket}[1]{\left\vert #1 \right\rangle}
\newcommand{\bra}[1]{\left\langle #1 \right\vert}
\newcommand{\braket}[2]{\left\langle #1 \middle\vert #2\right\rangle}
\newcommand{\commut}[1]{\left[#1\right]}
\newcommand{\de}{\mathrm{d}}
\newcommand{\dx}{\de x}
\newcommand{\tdev}[2]{\frac{\de #1}{\de #2}}
\newcommand{\pdev}[2]{\frac{\partial #1}{\partial #2}}
\newcommand{\tdevzero}[1]{\left.\tdev{}{#1}\right\vert_{#1 = 0}}
\newcommand{\set}[1]{\left\{#1\right\}}
% DOCUMENT INFO ----------
\title{Why is momentum the generator of translations?}
\author{Santiago Quintero de los Ríos\\{\small for homotopico.com}} \date{\today}

% BEGIN DOCUMENT ----------

\begin{document}
\maketitle

In quantum mechanics, it is often stated that \emph{the momentum [operator] is the generator of translations}. However, most texts fail to give a satisfactory justification to this claim, if any is given at all. In some cases, the proof is simply begging the question. In others, the burden of proof is shifted to classical mechanics, stating that ``this was already proved in classical mechanics''. Again, most standard texts of classical mechanics fail to give a clear proof of this claim, although the mathematical apparatus is there. What we will do here is define \emph{precisely} what it means for observable to be a generator of a group of transformations, both in quantum and classical mechanics, and prove that, indeed, the momentum observable \emph{is} the generator of translations in classical mechanics.


\section*{The origin of the question}
\label{sec:origin-question}

We define the translation operator $\hat{T}_a$ acting on elements $\psi\in L^2(\mathbb{R})$ (the Hilbert space of square-integrable complex functions
over $\mathbb{R}$) by,
of course, translating the wavefunction a distance $a$ to the right:
\begin{equation}
  (\hat{T}_a\psi)(x) := \psi(x-a).
\end{equation}
This operator is actually a whole family of operators that depend on the parameter $a$.
It is clear that $\hat{T}_0$ is the identity operator. Now since this family of operators
is particularly well-behaved\footnote{That is, it is a one-parameter absolutely continuous group.}, it has an
\textbf{infinitesimal generator}, which is some operator $\hat{w}$ such that
\begin{equation}
  \label{eq:1}
  \hat{T}_a = \exp\left(-ia\hat{w}\right).
\end{equation}
Yes, I know that the standard name for this generator is $\hat{p}$, but
I am trying to remove any possible meaning to this operator. For the time being, we only
know that $\hat{w}$ is the generator of translations. That's why I don't call it $\hat{p}$.

What does $\hat{w}$ look like? One way to calculate the infinitesimal generator of a family of operators is
by evaluating the derivative at $a=0$. If we expand the right-hand side of equation \ref{eq:1} as a power series, then
\begin{equation}
  \label{eq:3}
  \exp\left(-ia\hat{w}\right) = \sum_{n=0}^{\infty}\frac{1}{n!}(-ia\hat{w})^n,
\end{equation}
then it follows that
\begin{equation}
  \label{eq:4}
  \tdevzero{a}\hat{T}_a = \tdevzero{a} \exp\left(-ia\hat{w}\right) = -i\hat{w}.
\end{equation}
So let $\psi$ be any function on $L^2(\mathbb{R})$. We have that
\begin{equation}
  \tdevzero{a}\hat{T}_a\psi(x) = \tdevzero{a}\psi(x-a) = -\tdev{}{x}\psi(x),
\end{equation}
so we can say that the infinitesimal generator is
\begin{equation}
  \label{eq:5}
  \hat{w} = -i\tdev{}{x}.
\end{equation}
And hey, look at that. It turns out that this is what we call the momentum operator in quantum mechanics.

However, it is not clear \textbf{at all} why the operator $-i\tdev{}{x}$ should be given the highly suggestive name of ``momentum''. So
this it's not that the generator of translations is momentum because it has
the form $-i\tdev{}{x}$, but the other way around:

We say that $-i\tdev{}{x}$ is the momentum operator, because it is the one that generates translations.

Therefore, the question is shifted: \textbf{Why do we give the name ``momentum'' to the operator that generates translations?}

\section*{The usual answers}
\label{sec:first-answer}

If you only want a half-plausible argument, here it goes. Let $\hat{x}$ be the position operator, that is
the expectation value $\braket{\psi}{\hat{x}\psi}$ is the expected value for the position of a particle in
state $\psi$. It can be shown that the position operator acts on wavefunctions by multiplication by the variable
$x$, i.e.
\begin{equation}
  (\hat{x}\psi)(x)=x\psi(x).
\end{equation}
This equation can also be used to define $\hat{x}$. Now we want to compute the commutator $\commut{\hat{x},\hat{w}}$.
On one hand, we have:
\begin{equation}
  (\hat{x}\hat{w}\psi)(x) = -ix\tdev{}{x}\psi(x),
\end{equation}
but on the other hand,
\begin{equation}
  (\hat{w}\hat{x}\psi)(x) = -i\tdev{}{x}\left(x\psi(x)\right) = -i\left(\psi(x) + x\tdev{}{x}\psi(x)\right).
\end{equation}
Then, clearly
\begin{equation}
  \commut{\hat{x},\hat{w}}\psi(x) = i\psi(x),
\end{equation}
and so $\commut{\hat{x},\hat{w}}=i\hat{I}$. Now we consider the ``quantization rule'' that
turns \emph{classical} observables $f$ into Hermitian operators $\hat{f}$ and that turns Poisson brackets
\emph{almost} into Lie brackets 
\begin{equation}
  \set{f,g} \mapsto i\commut{\hat{f},\hat{g}},
\end{equation}
and what we do is go backwards! We found an operator $\hat{w}$ that satisfies the commutation relation
$\commut{\hat{x},\hat{w}}=i\hat{I}$, so the \emph{classical} observables they come from satisfy the Poisson-bracket
relation $\set{x,w}=1$. We know that $\hat{x}$ is the position operator, so $x$ must be the position observable. On the other
hand, we know that for the canonical momentum $p$,
\begin{equation}
  \set{x,p} = \pdev{x}{x} \pdev{p}{p}-\pdev{x}{p} \pdev{p}{x} = 1.
\end{equation}
We can clearly see that $\set{x,p}=\set{x,w}=1$, so this suggests that $w=p$! And we're done. $w$ is the momentum, so the operator
$\hat{w}$ is the momentum operator.

\textbf{But wait a minute...}

The relation
\begin{equation}
  \set{x,w} = 1
\end{equation}
is not enough to \emph{define} $w$! This is because the operator $f\mapsto\set{x,f}$ is degenerate. As a trivial example,
\begin{equation}
  \set{x,p+x} = 1,
\end{equation}
and actually, for any function $f(x)$ that \emph{does not depend on }$p$, it follows that
\begin{equation}
  \set{x,p+f(x)}=1.
\end{equation}
This means that $w$ is an observable of the form $ w = p + f(x)$ for \emph{some} function $f(x)$.
At this point you might think ``eh, close enough''. And fair enough, then you're done!

But I'm not thoroughly (actually, not at all) satisfied. There is another clue though. This previous
argument is one of two that I've seen in QM books. The other one is as follows:
\begin{center}
  \textit{In classical mechanics it is shown that momentum is the generator of
    translations.}
\end{center}
And well, yes, that could be enough for our purposes. We can't expect
every new theory to work out of the box and be able to stand by itself, right? When
we construct a new theory from a mathematical perspective, we look back to
other theories that are well-established and we understand, look for analogies and
similarities, and interpret the symbols in such a way that everything is as consistent
as possible with what came before.

Therefore it is quite reasonable to say that this operator $\hat{w}$ is the momentum operator if
we already know from classical mechanics that the generator of translations is the momentum. We've shown that
in quantum mechanics, translations \emph{also} have infinitesimal generators, so it is natural
to interpret those as momenta.

Again, that is \textbf{if} we already know from classical mechanics that momentum is the generator of translations.
And I think I missed that lecture? That's a big ``if''.

That quote up there shifts the burden of proof to classical mechanics. So what we're going to do is plunge into classical mechanics
and try to show that momentum is the generator of translations from two different points of view: the ``standard'' Goldstein point-of-view, and the
more modern symplectic geometry point-of-view. We will see from both that we can reasonably show that, in a certain way, momentum
is the generator of translations.

\section*{The standard point of view}
\label{sec:standard-pov}

More precisely, what we will prove is the following:\\
To each (classical) observable $f$ we can \emph{canonically} assign a one-parameter group
of \emph{canonical} transformations $\Phi_a$ that preserve $f$. In this case, we say that $f$ generates $\Phi_a$. When we choose the canonical momentum $f=p$, the
resulting transformations are translations, and thus we say that $p$ is the generator of translations.

In this first section, we will do it in a bit of a dirty way. If infinitesimals make you uncomfortable (completely understandable), then
always note that when we write, e.g. $\tilde{x} = x + \epsilon X$, what we mean is that $\tilde{x}=\tilde{x}(\epsilon)$ is a function of $\epsilon$, that $\tilde{x}(0)=x$, and $X = \lim_{\epsilon\to 0}\frac{1}{\epsilon}\tilde{x}(\epsilon)$. If you would like to see a more concise proof that requires a bit of differential topology, skip to the next section.

Consider some system with phase space $\mathcal{S}$, with canonical coordinates $q^i,p_j$, and let $f$ be an arbitrary observable. We want to find a \emph{canonical} transformation\footnote{And an additional question that I would like to explore is ``what's the huge deal about canonical transformations''? We know that they preserve the structure of the Hamilton-Jacobi equations, but not necessarily the Hamiltonian, so that doesn't seem like much, does it? Well it turns out that the \emph{structure} itself of the equations has an interesting and beautiful mathematical background... But I'm saving that for later. }
$Q^i = Q^i(q,p)$, $P_j = P_j(q,p)$ that preserves $f$ and depends on only one parameter $\epsilon$. For small $\epsilon$, we can write
\begin{equation}
  \begin{aligned}
    Q^i(\epsilon) &= q^i + \epsilon A^i\\
    P_j(\epsilon) &= p_j + \epsilon B_j
  \end{aligned},
\end{equation}
with $A^i,B_j$ functions of $q,p$ that we want to determine. Since we want these transformations to be canonical, the Poisson brackets must be conserved, so
\begin{equation}
  \begin{aligned}
    \set{q^i,p_i} = 1 &= \set{Q^i,P_i} \\
    &= \set{q^i + \epsilon A^i,p_i + \epsilon B_i}\\
    &= \set{q^i,p_i} + \epsilon\left(\set{A^i,p_i}+\set{q^i,B_i}\right) + \mathcal{O}(\epsilon^2).
  \end{aligned}
\end{equation}
Then, working up to first order in $\epsilon$, we require that
\begin{equation}
  \set{A^i,p_i}+\set{q^i,B_i} = 0.
\end{equation}
Unravel these Poisson brackets,
\begin{equation}
  \set{A^i,p_i} = \sum_k\left(\pdev{A^i}{q^k}\pdev{p_i}{p_k} - \pdev{A^i}{p_k}\pdev{p_i}{q^k}\right) = \pdev{A^i}{q^i}
\end{equation}
\begin{equation}
  \set{q^i,B_i} = \sum_k\left(\pdev{q^i}{q^k}\pdev{B_i}{p_k} - \pdev{q^i}{p_k}\pdev{B_i}{q^k}\right) = \pdev{B_i}{p_i},
\end{equation}
then we require $A^i,B_i$ to be such that
\begin{equation}
  \label{eq:2}
  \pdev{A^i}{q^i} + \pdev{B_i}{p_i} = 0.
\end{equation}
One way to guarantee that condition \eqref{eq:2} is satisfied is by finding some \emph{other} observable, say $g$, and write
\begin{equation}
  \label{eq:7}
  \begin{aligned}
    A^i &= \pdev{g}{p_i}\\
    B_j &= -\pdev{g}{q^j}
  \end{aligned}.
\end{equation}
However, not every $g$ will work, since we also require the transformation to preserve the observable $f$. So again, for small
$\epsilon$, we can write
\begin{equation}
  \begin{aligned}
    f\left(Q^i,P_j \right) &= f\left(q^i+\epsilon A^i,p_j+\epsilon B_j\right)\\
    &= f\left(q^i,p_j\right) + \epsilon\sum_k\left(A^k\pdev{f}{q^k} + B_k\pdev{f}{p_k}\right) + \mathcal{O}(\epsilon^2).
\end{aligned}
\end{equation}
Since we require the transformation to preserve $f$, then $f\left(Q^i,P_j \right) = f(q^i,p_j)$ and thus
\begin{equation}
  \label{eq:6}
  \sum_k\left(A^k\pdev{f}{q^k} + B_k\pdev{f}{p_k}\right) = 0.
\end{equation}
But again, if we assume that $A^i,B_j$ come from an observable $g$ as in equation \eqref{eq:7}, this condition
becomes
\begin{equation}
  \label{eq:6}
  \sum_k\left(\pdev{g}{p^k}\pdev{f}{q^k} - \pdev{g}{q^k}\pdev{f}{p_k}\right) = \set{f,g}=0.
\end{equation}
Therefore the admissible observables $g$ that can be used for equation \eqref{eq:7} are such that $\set{f,g}=0$.
Which one is the \emph{absolute simplest one} to choose? Well $f$ itself! This works since $\set{f,f}=0$.
Therefore the \emph{natural} choice is $g=f$, so that the transformation \emph{for infinitesimal} $\epsilon$ is
\begin{equation}
  \label{eq:8}
  \begin{aligned}
    Q^i(\epsilon) &= q^i + \epsilon \pdev{f}{p^i}\\
    P_j(\epsilon) &= p_j - \epsilon \pdev{f}{q^j}
  \end{aligned}.
\end{equation}
This equation only holds for infinitesimally small $\epsilon$, since we've been working only up to first order. What we want now is to write the exact solution  to large $\epsilon$. To do so, note that equation~\eqref{eq:8}
gives us a \emph{differential equation} for $Q$ and $P$: writing $q^i=Q^i(0), p_j=P_j(0)$, it can be easily seen that
\begin{equation}
  \label{eq:13}
  \begin{aligned}
    \tdev{Q^i}{\epsilon} &= \pdev{f}{p^i}\\
    \tdev{P_j}{\epsilon} &= -\pdev{f}{q^j}.
  \end{aligned}
\end{equation}
Note that these are just like the Hamilton-Jacobi equations, except that ``time'' is $\epsilon$ and the ``Hamiltonian''
is $f$!

Now we see how any general observable
changes when we do an infinitesimal transformation \eqref{eq:8}. Let $g$ be any observable. Then,
if we start at a point $(q_0,p_0)$, and write $(q_1,p_1)=(Q(\epsilon),P(\epsilon))$ (I'm dropping the $i,j$ indices), then
\begin{equation}
  \label{eq:9}
    \begin{aligned}
    g\left(Q^i(\epsilon),P_j(\epsilon) \right) &= g\left(q_0^i+\epsilon \left.\pdev{f}{p_{j}}\right|_{(q_0,p_0)},p_{0,j}-\left.\pdev{f}{q^j}\right|_{(q_0,p_0)}\right)\\
    g(q_1,p_1)&= g\left(q_0,p_{0}\right) + \epsilon\sum_k\left(\left.\pdev{f}{p_{k}}\right|_{(q_0,p_0)}\left.\pdev{g}{q^k}\right|_{(q_0,p_0)} - \left.\pdev{f}{q^k}\right|_{(q_0,p_0)}\left.\pdev{g}{p_k}\right|_{(q_0,p_0)}\right) + \mathcal{O}(\epsilon^2)\\
    g(q_1,p_1)&= g\left(q_0,p_{0}\right) + \epsilon\left.\set{g,f}\right|_{(q_0,p_0)} +\mathcal{O}(\epsilon^2).
\end{aligned}
\end{equation}
How do we go to large $\epsilon$? Well, the last equation tells us that
\begin{equation}
  \tdev{}{\epsilon}g\left(Q(\epsilon),P(\epsilon)\right) = \set{g,f}_{Q(\epsilon),P(\epsilon)},
\end{equation}
therefore, if we write $g(\epsilon) = g(Q(\epsilon),P(\epsilon))$, taking the derivative with respect to $\epsilon$ again we obtain:
\begin{equation}
  \tdev{^2}{\epsilon^2}g\left(\epsilon\right) = \tdev{}{\epsilon}\set{g,f}_\epsilon = \set{\set{g,f},f}_\epsilon.
\end{equation}
So if we call $X_f$ the differential operator $X_f(g):=\set{g,f}$, we can see that
\begin{equation}
  \tdev{^n}{\epsilon^n}g\left(\epsilon\right) = X_f(X_f(\dots X_f(g)))_{\epsilon}={X_f}^n(g)_{\epsilon}.
\end{equation}
And thus, by Taylor-expanding $g$ with respect to $\epsilon$ around $\epsilon=0$, we see that
\begin{equation}
  \label{eq:12}
  g(\epsilon) = \sum_{n=0}^{\infty}\frac{1}{n!}\left.\tdev{^n}{\epsilon^n}\right|_{\epsilon=0}g(\epsilon) = \sum_{n=0}^{\infty}\frac{1}{n!}{X_f}^n(g)_{(q_0,p_0)} = \exp\left(\epsilon {X_f}(0)\right)g.
\end{equation}
We call $X_f$ the \textbf{infinitesimal generator} associated to $f$.

Alright. We should stop here for a bit and gather what we have. We have shown that for any observable $f$, we can find a family of
canonical transformations, which we write as $Q(\epsilon),P(\epsilon)$, given by the set of Hamilton-Jacobi-like equations~\eqref{eq:13}.
Note that if we fix a starting point $(q_0,p_0)$, this family of transformations gives rise to a \emph{curve} in phase space, given by
$q= Q(\epsilon),p=P(\epsilon)$.
Any other observable $g$ also changes along this curve as $g(\epsilon) = \exp(\epsilon X_f(0))g$, or rather as $\tdev{g}{\epsilon}=\set{g,f}$.
In particular, $f$ is constant along this curve, since $\tdev{}{\epsilon}f = \set{f,f}=0$. Therefore the family of transformations is
canonical and preserves $f$, as was desired.

Now here comes what was promised. If we choose $f=p_j$, the momentum observable, then the infinitesimal
generator associated to $p_j$, $X_{p_j}$, is the differential operator
\begin{equation}
  X_{p_j}(g) = \set{g,p_j} = \sum_{k}\pdev{g}{q^k}\pdev{p_j}{p_k} - \pdev{g}{p_k}\pdev{p_j}{q^k} = \pdev{}{q^j}(g).
\end{equation}
Ring any bells?

Furthermore, the family of transformations is the solution to the Hamilton-Jacobi-like equations~\eqref{eq:13} with $f=p_j$, i.e.
\begin{equation}
  \label{eq:14}
  \begin{aligned}
    \tdev{Q^i}{\epsilon} &= \pdev{p_j}{p^i} = \delta^i_j\\
    \tdev{P_k}{\epsilon} &= -\pdev{p_j}{q^k} = 0,
  \end{aligned}
\end{equation}
which is readily integrated, given an initial point $q_0 = Q(0),p_0=P(0)$:
\begin{equation}
  Q^j(\epsilon) = q_{0}^j + \epsilon;
\end{equation}
and all the other coordinates constant. This is indeed a translation in the $j$-th space coordinate.

So there you have it! In classical mechanics, to each observable $f$ we assign a differential operator $X_f$ that
generates one-parameter canonical transformations that preserve $f$. In the case where $f=p$, the differential operator is $X_p = \pdev{}{q}$,
and the canonical transformations are translations! Therefore we say that \textbf{momentum is the generator of translations}.

\section*{The mathematician's way}
\label{sec:mathematicians-way}

This way is basically exactly the same as the one before, just with the more sophisticated language
of differential topology, specifically in the context of symplectic manifolds. It can be easily seen
that the results given here are exactly the same as the results given in the previous section, when
we write them in Darboux coordinates.

So first, some definitions.

A \textbf{symplectic manifold} is a smooth manifold $\mathcal{M}$ of dimension $2n$ endowed with a closed, non-degenerate $2$-form $\omega$, which we call a \textbf{symplectic form}. It can be shown that for each point
$x\in \mathcal{M}$ there exists a neighborhood $U$ of $x$ and coordinates $q^1,\dots,q^n,p_1,\dots,p_n$ such
that the symplectic form can be written as
\begin{equation}
  \label{eq:11}
  \omega = \de q^{\mu}\wedge\de p_{\mu}.
\end{equation}
Here we are using Einstein's sum notation. This result is \textbf{Darboux's theorem}, and the coordinates $(q^\mu,p_\mu)$ are called \textbf{Darboux coordinates}.


Since
$\omega$ is non-degenerate, it induces an isomorphism $(-)_\flat:T_p\mathcal{M}\to T_p^*\mathcal{M}$, given
pointwise by the relation
\begin{equation}
  \label{eq:10}
  v_\flat(u):=\omega(v,u)
\end{equation}
for all $u,v\in T_p\mathcal{M}$ and all $p\in\mathcal{M}$. The inverse of $(-)_\flat$ is denoted
by $(-)^\sharp:T_p^*\mathcal{M}\to T_p\mathcal{M}$.

This is basically all we need! Let $f\in C^{\infty}(\mathcal{M})$. Then we can define the \textbf{Hamiltonian vector field associated to $f$}, denoted by $X_f\in\mathfrak{X}(\mathcal{M})$, as
\begin{equation}
  \label{eq:15}
  X_f = (\de f)^{\sharp}.
\end{equation}
By this definition, for any other vector field $Y$, we have that
\begin{equation}
  \label{eq:16}
  \omega(X_f,Y) = \de f(Y)= Y[f].
\end{equation}
Therefore an alternative definition for $X_f$ is the unique vector field such that $\iota_{X_f}\omega = \de f$.

Using Cartan's magic formula, we see that
\begin{equation}
  \label{eq:17}
  L_{X_f}\omega = \iota_{X_f}(\de\omega) + \de\left(\iota_{X_f}\omega\right) = 0.
\end{equation}
Here, the first term is zero since $\omega$ is closed (i.e. $\de\omega=0$), and the second one is zero too since $\iota_{X_f}\omega=\de f$, and $\de^2=0$. In addition to this, the directional derivative of $f$ along $X_f$ is zero, since
\begin{equation}
  \label{eq:19}
  X_f(f) = \de f(X_f) = \iota_{X_f}\omega(X_f)=\omega(X_f,X_f)=0.
\end{equation}
This means that if $\Phi^f_\tau$ is the flow of $X_f$, then
for all values of $\tau$
\begin{equation}
  \label{eq:18}
  \left(\Phi^f_{\tau}\right)^*\omega = \omega,
\end{equation}
and $f\circ\Phi^f_\tau = f$. That is, neither $f$ nor $\omega$ change along the integral curves of the Hamiltonian vector field associated to $f$.

Furthermore, since $\Phi^f_{\tau}:\mathcal{M}\to\mathcal{M}$ is a diffeomorphism, then it satisfies the conditions for being a \textbf{symplectomorphism} or \textbf{canonical transformation}. Namely, a diffeomorphism $f:\mathcal{M}\to\mathcal{N}$ between two symplectic manifolds $(\mathcal{M},\omega)$ and $(\mathcal{N},\Omega)$ is a symplectomorphism if $f^*\Omega=\omega$.

What we have now is the following: to any $f\in C^\infty(\mathcal{M})$ we can assign a Hamiltonian vector field $X_f\in \mathfrak{X}(\mathcal{M})$ whose flow $\Phi^f_\tau$ is a symplectomorphism that preserves $f$. Now if we work locally in Darboux coordinates $(q^\mu,p_\mu)$, so that $\omega = \de q^\mu\wedge \de p_\mu$. If
we write $X_f$ locally as
\begin{equation}
  \label{eq:20}
  X_f = A^\mu\pdev{}{q^\mu} + B_{\nu}\pdev{}{p_\nu},
\end{equation}
then we have that
\begin{equation}
  \label{eq:21}
  \begin{aligned}
    \iota_{X_f}\omega &= \iota_{X_f}\left(\de q^\mu\wedge \de p_\mu\right)\\
    &= \de q^\mu(X_f)\de p_\mu - \de q^\mu\de p_\mu(X_f)\\
    &= A^\mu\de p_\mu - B_\mu\de q^\mu = \de f.
  \end{aligned}
\end{equation}
However, since
\begin{equation*}
  \de f = \pdev{f}{q^\mu}\de q^\mu + \pdev{f}{p_\mu}\de p_\mu,
\end{equation*}
then necessarily the components of $X_f$ must be
\begin{equation}
  \label{eq:22}
  \begin{aligned}
    A^\mu &= \pdev{f}{p_\mu}\\
    B_\mu &= -\pdev{f}{q^\mu}\\
  \end{aligned}.
\end{equation}
The Hamiltonian vector field associated to $f$ is then, in Darboux coordinates,
\begin{equation}
  \label{eq:23}
  X_f = \pdev{f}{p_\mu}\pdev{}{q^\mu} -\pdev{f}{q^\mu}\pdev{}{p_\mu} = \set{-,f}.
\end{equation}

And once again, if we choose $p_\nu =f$, then the Hamiltonian vector field is
\begin{equation}
  \label{eq:24}
  X_{p_\nu} = \pdev{}{q^\nu},
\end{equation}
So that the flow of $X_{p_\nu}$ is simply translation along the $q^\nu$ coordinate. Once again, we can
say that the [Hamiltonian vector field associated to the] canonical momentum is the generator of translations.

\section*{The takeaway}
\label{sec:takeaway}

We proved a general result in classical mechanics: to any observable $f$, we can assign a differential
operator $X_f=\set{-,f}$, called the infinitesimal generator or Hamiltonian vector field associated to $f$,
that \emph{generates} canonical transformations that preserve $f$. The one-parameter group of
canonical transformations associated to $X_f$ is given by $\Phi^f_\tau = \exp(\tau X_f)$, and we
say that $f$ is the generator of $\Phi^f$.

In particular, if we choose the canonical momentum $p$ to be the observable $f$, then the infinitesimal generator is $X_p = \tdev{}{q}$ and the group of canonical transformations associated to $X_p$ is precisely the group of translations of the $q$ coordinate.
Therefore, we can say that \textbf{the momentum is the generator of translations}.

Now in quantum mechanics, we simply \textbf{define} the momentum operator $\hat{p}$ to be the infinitesimal
generator of the unitary group of translations, and we see that this definition is consistent with canonical
quantization.

\subsection*{References}
\label{sec:references}

\begin{itemize}
\item Sakurai, J. J. (1995). \textit{Modern Quantum Mechanics, Revised Edition}.
\item Goldstein, H. , Poole, C. \& Safko, J. (2001). \textit{Classical Mechanics, Third Edition}.
\item José, J. V. \& Saletan, E. (1998). \textit{Classical Dynamics: A Contemporary Approach}. This book is great. I love it. Read it.
\item Abraham, R. \& Marsden, J. (2008). \textit{Foundations of Mechanics}.
\end{itemize}
Great thanks to Laura Arboleda for checking style and consistency.
\end{document}

% END DOCUMENT ----------


%%% Local Variables:
%%% mode: latex
%%% TeX-master: t
%%% End:
